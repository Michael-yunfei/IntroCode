% This is the test file for LaTeX

\documentclass[12pt]{article}

% add some essential packages, some might not be used

\usepackage[T1]{fontenc}
\usepackage[utf8]{inputenc}
\usepackage[usenames,dvipsnames]{color}
\usepackage{natbib}
\usepackage{authblk}
\usepackage{ragged2e}
\usepackage{amsmath}
\usepackage[a4paper,margin=1in,bottom=1.0in]{geometry}
\usepackage{url}
\usepackage{array}
\usepackage{bbding}
\usepackage{amssymb}
\usepackage{graphicx}  % mini page function
\usepackage{adjustbox}
\usepackage{subcaption}
\usepackage{booktabs}
\usepackage{float}
\usepackage{appendix} % appendix package
\usepackage{hyperref}
\usepackage{url}
\usepackage[english]{babel}
\usepackage{adjustbox}
\usepackage{enumitem}
\usepackage{textgreek}
\usepackage{soul}  % package for highlighting
\usepackage{listings}
\usepackage{wasysym}
\usepackage{amsthm}
\usepackage{framed}
\usepackage{bm}
\usepackage{booktabs}  % package for table line


\usepackage{rotating} % for the horizontal page table

\usepackage{tikz}
\usetikzlibrary{calc}
\usetikzlibrary{matrix}
\usetikzlibrary{positioning}
\usepackage{color}
\usepackage{setspace}
\usepackage{xcolor}  % shaded text with defined color

\usepackage{tcolorbox} % package for making colorful box

 \setlength{\parskip}{0.15cm} % change the paragraph spacing
\renewcommand\labelitemi{$\vcenter{\hbox{\tiny$\bullet$}}$} % set the bullet size as tiny

% \newcommand*\rot{\rotatebox{90}} % for rotate text

\usepackage{sectsty} %package for section size

\sectionfont{\fontsize{14}{12}\selectfont} % Change the section font size

\subsectionfont{\fontsize{13}{12}\selectfont}
\subsubsectionfont{\fontsize{12}{12}\selectfont}

\newcommand\numberthis{\addtocounter{equation}{1}\tag{\theequation}} % new command



\theoremstyle{definition}
\newtheorem{definition}[subsubsection]{Definition}
\newtheorem{axiom}[subsection]{Axiom}
\newtheorem{example}[subsubsection]{Example}
\newtheorem{theorem}[subsubsection]{Theorem}
\newtheorem{proposition}[subsubsection]{Proposition}

\usepackage{courier}

% tikzsetting

\usetikzlibrary{shapes,decorations,arrows,calc,arrows.meta,fit,positioning}

\tikzset{
    -Latex, auto, node distance =1 cm and 1 cm, semithick,
    state/.style ={ellipse, draw, minimum width = 0.7 cm},
    point/.style = {circle, draw, inner sep=0.04cm, fill, node contents={}},
    bidirected/.style={Latex-Latex,dashed},
    el/.style = {inner sep=2pt, align=left, sloped}
}


% Define colors
\definecolor{cmd}{HTML}{F7F7F9}

\begin{document}

\title{Build up Latex in Atom: packages and tips }
\author{Michael}
\date{April, 2019}


\maketitle  % always use \maketitle in the end


\section{Introduction}

At the very beginning, it takes me a while to build up the collective IDE for \texttt{Python, R , LaTex} in Atom. This document records the procedure to do this. I will collect the common problems any one might encounter when using Atom to write a paper or do a project with \texttt{Python, R} and \texttt{LaTex}.


\section{Packages for LaTex in Atom}

To build up IDE for LaTex in Atom, you need the following packages:
\begin{itemize}
  \item \texttt{atom-latex}: type LaTex code and compile in Atom
  \item \texttt{language-latex}: syntax highlighting
  \item \texttt{latex-autcomplete}: LaTex code autocomplete
  \item \texttt{linter-spell-lates}: spell checking
  \item \texttt{pdf-view}: view pdf in Atom
\end{itemize}

\section{Tips for LaTex in Atom}

During setting up my LaTex environment in Atom, I had several main problems which annoyed me. The fist one is the spell-check does not work at the very beginning. To solve this problem, you need add the source file into the \texttt{Grammars} of \texttt{spell-check} package. To check your source file, follow those steps:
\begin{enumerate}
  \item use \colorbox{cmd}{\texttt{command-shift-p}} to open the command window
  \item Then type \colorbox{cmd}{\texttt{Editor:Log Cursor Scope}} there, your file source will show up
  \item Copy the source format, such as \colorbox{cmd}{\texttt{text.tex.latex}}, paste it into \colorbox{cmd}{\texttt{Grammars}}
\end{enumerate}

\textbf{Warning}: in order to make sure the spell-check works, you need make sure the file grammar you are using is `LaTex'. Otherwise, the spell-check will not work.

To make the compiling clean and organized, you also need set the deleing file behaviors in preference part. 

\section{Tips for Python in Atom}


\section{Tips for R in Atom }




















\end{document}
