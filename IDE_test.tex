\documentclass[12pt]{article}
% add some essential packages, some might not be used

\usepackage[T1]{fontenc}
\usepackage[utf8]{inputenc}
\usepackage[usenames,dvipsnames]{color}
\usepackage{natbib}
\usepackage{authblk}
\usepackage{ragged2e}
\usepackage{amsmath}
\usepackage[a4paper,margin=1in,bottom=1.0in]{geometry}
\usepackage{url}
\usepackage{array}
\usepackage{bbding}
\usepackage{amssymb}
\usepackage{graphicx}  % mini page function
\usepackage{adjustbox}
\usepackage{subcaption}
\usepackage{booktabs}
\usepackage{float}
\usepackage{appendix} % appendix package
\usepackage{hyperref}
\usepackage{url}
\usepackage[english]{babel}
\usepackage{adjustbox}
\usepackage{enumitem}
\usepackage{textgreek}

\usepackage{listings}
\usepackage{wasysym}
\usepackage{amsthm}
\usepackage{framed}
\usepackage{bm}
\usepackage{booktabs}  % package for table line


\usepackage{rotating} % for the horizontal page table

\usepackage{tikz}
\usetikzlibrary{calc}
\usetikzlibrary{matrix}
\usetikzlibrary{positioning}
\usepackage{color}
\usepackage{setspace}
\usepackage{xcolor}

\usepackage{tcolorbox} % package for making colorful box

 \setlength{\parskip}{0.15cm} % change the paragraph spacing
\renewcommand\labelitemi{$\vcenter{\hbox{\tiny$\bullet$}}$} % set the bullet size as tiny

% \newcommand*\rot{\rotatebox{90}} % for rotate text

\usepackage{sectsty} %package for section size

\sectionfont{\fontsize{14}{12}\selectfont} % Change the section font size

\subsectionfont{\fontsize{13}{12}\selectfont}
\subsubsectionfont{\fontsize{12}{12}\selectfont}

\newcommand\numberthis{\addtocounter{equation}{1}\tag{\theequation}} % new command



\theoremstyle{definition}
\newtheorem{definition}[subsubsection]{Definition}
\newtheorem{axiom}[subsection]{Axiom}
\newtheorem{example}[subsubsection]{Example}
\newtheorem{theorem}[subsubsection]{Theorem}
\newtheorem{proposition}[subsubsection]{Proposition}

\usepackage{courier}

% tikzsetting

\usetikzlibrary{shapes,decorations,arrows,calc,arrows.meta,fit,positioning}

\tikzset{
    -Latex,auto,node distance =1 cm and 1 cm,semithick,
    state/.style ={ellipse, draw, minimum width = 0.7 cm},
    point/.style = {circle, draw, inner sep=0.04cm,fill,node contents={}},
    bidirected/.style={Latex-Latex,dashed},
    el/.style = {inner sep=2pt, align=left, sloped}
}

\lstset{language=Matlab}

\definecolor{mygreen}{rgb}{0,0.6,0}
\definecolor{mygray}{rgb}{0.5,0.5,0.5}
\definecolor{mymauve}{rgb}{0.58,0,0.82}

\lstset{
  backgroundcolor=\color{white},   % choose the background color; you must add \usepackage{color} or \usepackage{xcolor}; should come as last argument
  basicstyle=\footnotesize,        % the size of the fonts that are used for the code
  breakatwhitespace=false,         % sets if automatic breaks should only happen at whitespace
  breaklines=true,                 % sets automatic line breaking
  captionpos=b,                    % sets the caption-position to bottom
  commentstyle=\color{mygreen},    % comment style
  deletekeywords={...},            % if you want to delete keywords from the given language
  escapeinside={\%*}{*)},          % if you want to add LaTeX within your code
  extendedchars=true,              % lets you use non-ASCII characters; for 8-bits encodings only, does not work with UTF-8
  frame=single,	                   % adds a frame around the code
  keepspaces=true,                 % keeps spaces in text, useful for keeping indentation of code (possibly needs columns=flexible)
  keywordstyle=\color{blue},       % keyword style
  language=Matlab,                 % the language of the code
  morekeywords={*,...},            % if you want to add more keywords to the set
  numbers=left,                    % where to put the line-numbers; possible values are (none, left, right)
  numbersep=5pt,                   % how far the line-numbers are from the code
  numberstyle=\tiny\color{mygray}, % the style that is used for the line-numbers
  rulecolor=\color{black},         % if not set, the frame-color may be changed on line-breaks within not-black text (e.g. comments (green here))
  showspaces=false,                % show spaces everywhere adding particular underscores; it overrides 'showstringspaces'
  showstringspaces=false,          % underline spaces within strings only
  showtabs=false,                  % show tabs within strings adding particular underscores
  stepnumber=2,                    % the step between two line-numbers. If it's 1, each line will be numbered
  stringstyle=\color{mymauve},     % string literal style
  tabsize=2,	                   % sets default tabsize to 2 spaces
  title=\lstname                   % show the filename of files included with \lstinputlisting; also try caption instead of title
}

\numberwithin{equation}{section}
\numberwithin{figure}{section}
\numberwithin{table}{section}

% Define colors
\definecolor{cmd}{HTML}{F7F7F9}

\begin{document}

\title{Build up Latex in Atom: packages and tips }
\author{Michael}
\date{April, 2019}


\maketitle  % always use \maketitle in the end


\section{Introduction}

At the very beginning, it takes me a while to build up the collective IDE for \texttt{Python, R , LaTex} in Atom. This document records the procedure to do this. I will collect the common problems any one might encounter when using Atom to write a paper or do a project with \texttt{Python, R} and \texttt{LaTex}.


\section{Packages for LaTex in Atom}

To build up IDE for LaTex in Atom, you need the following packages:
\begin{itemize}
  \item \texttt{atom-latex}: type LaTex code and compile in Atom
  \item \texttt{language-latex}: syntax highlighting
  \item \texttt{latex-autcomplete}: LaTex code autocomplete
  \item \texttt{linter-spell-lates}: spell checking
  \item \texttt{pdf-view}: view pdf in Atom
\end{itemize}

\section{Tips for LaTex in Atom}

During setting up my LaTex environment in Atom, I had several main problems which annoyed me. The fist one is the spell-check does not work at the very beginning. To solve this problem, you need add the source file into the \texttt{Grammars} of \texttt{spell-check} package. To check your source file, follow those steps:
\begin{enumerate}
  \item use \colorbox{cmd}{\texttt{command-shift-p}} to open the command window
  \item Then type \colorbox{cmd}{\texttt{Editor:Log Cursor Scope}} there, your file source will show up
  \item Copy the source format, such as \colorbox{cmd}{\texttt{text.tex.latex}}, paste it into \colorbox{cmd}{\texttt{Grammars}}
\end{enumerate}

\textbf{Warning}: in order to make sure the spell-check works, you need make sure the file grammar you are using is `LaTex'. Otherwise, the spell-check will not work.

To make the compiling clean and organized, you also need set the deleting file behaviors in preference part.

If you put repeated bib entry in your BiBtext file, there will be a compiling error.

Error: Paragraph ended before \textbackslash BR@@bibitem was complete.\textbackslash par

Solution: put \textbackslash in front of \% in your url links

Error: latex error code 12

Solution: bibliographystyle should use apalike or something

If you use different typing system, for example, in my case I used Chinese typing system to type some english words, then you will have the following errors:
\textit{Package inputenc Error: Unicode character} H (U+8).

\textbf{Special characters}: The following characters play a special role in LaTeX and are called ``special printing characters", or simply ``special characters":

\# \$ \% \&  \_ \^ $\{  \}$ $\backslash$

Whenever you put one of these special characters into your file, you are doing something special. If you simply want the character to be printed just as any other letter, include a $\backslash$ in front of the character. For example, $\backslash$ \$ will produce \$ in your output. The exception to the rule is the $\backslash$ itself because $\backslash \backslash$ has its own special meaning. A $\backslash$ is produced by typing $\backslash$ in your file.


You can use $\backslash$phantom to hide some parts of latex


\textbf{Always use apalike for bibliographystyle}

\section{Tips for Python in Atom}



\section{Tips for R in Atom }




















\end{document}
